%% Generated by Sphinx.
\def\sphinxdocclass{report}
\documentclass[letterpaper,10pt,english]{sphinxmanual}
\ifdefined\pdfpxdimen
   \let\sphinxpxdimen\pdfpxdimen\else\newdimen\sphinxpxdimen
\fi \sphinxpxdimen=.75bp\relax
\ifdefined\pdfimageresolution
    \pdfimageresolution= \numexpr \dimexpr1in\relax/\sphinxpxdimen\relax
\fi
%% let collapsible pdf bookmarks panel have high depth per default
\PassOptionsToPackage{bookmarksdepth=5}{hyperref}
%% turn off hyperref patch of \index as sphinx.xdy xindy module takes care of
%% suitable \hyperpage mark-up, working around hyperref-xindy incompatibility
\PassOptionsToPackage{hyperindex=false}{hyperref}
%% memoir class requires extra handling
\makeatletter\@ifclassloaded{memoir}
{\ifdefined\memhyperindexfalse\memhyperindexfalse\fi}{}\makeatother

\PassOptionsToPackage{booktabs}{sphinx}
\PassOptionsToPackage{colorrows}{sphinx}

\PassOptionsToPackage{warn}{textcomp}

\catcode`^^^^00a0\active\protected\def^^^^00a0{\leavevmode\nobreak\ }
\usepackage{cmap}
\usepackage{xeCJK}
\usepackage{amsmath,amssymb,amstext}
\usepackage{babel}



\setmainfont{FreeSerif}[
  Extension      = .otf,
  UprightFont    = *,
  ItalicFont     = *Italic,
  BoldFont       = *Bold,
  BoldItalicFont = *BoldItalic
]
\setsansfont{FreeSans}[
  Extension      = .otf,
  UprightFont    = *,
  ItalicFont     = *Oblique,
  BoldFont       = *Bold,
  BoldItalicFont = *BoldOblique,
]
\setmonofont{FreeMono}[
  Extension      = .otf,
  UprightFont    = *,
  ItalicFont     = *Oblique,
  BoldFont       = *Bold,
  BoldItalicFont = *BoldOblique,
]



\usepackage[Sonny]{fncychap}
\ChNameVar{\Large\normalfont\sffamily}
\ChTitleVar{\Large\normalfont\sffamily}
\usepackage{sphinx}

\fvset{fontsize=\small,formatcom=\xeCJKVerbAddon}
\usepackage{geometry}


% Include hyperref last.
\usepackage{hyperref}
% Fix anchor placement for figures with captions.
\usepackage{hypcap}% it must be loaded after hyperref.
% Set up styles of URL: it should be placed after hyperref.
\urlstyle{same}


\usepackage{sphinxmessages}




\title{学习笔记}
\date{2023 年 09 月 05 日}
\release{0.1}
\author{Yancy}
\newcommand{\sphinxlogo}{\vbox{}}
\renewcommand{\releasename}{发行版本}
\makeindex
\begin{document}

\ifdefined\shorthandoff
  \ifnum\catcode`\=\string=\active\shorthandoff{=}\fi
  \ifnum\catcode`\"=\active\shorthandoff{"}\fi
\fi

\pagestyle{empty}
\sphinxmaketitle
\pagestyle{plain}
\sphinxtableofcontents
\pagestyle{normal}
\phantomsection\label{\detokenize{index::doc}}


\sphinxstepscope


\chapter{目录}
\label{\detokenize{list:id1}}\label{\detokenize{list::doc}}

\section{数值分析}
\label{\detokenize{list:id2}}

\subsection{数值分析第一章}
\label{\detokenize{list:id3}}
\sphinxAtStartPar
详见{\hyperref[\detokenize{files/Chap1::doc}]{\sphinxcrossref{\DUrole{std,std-doc}{此处}}}}


\subsection{数值分析第二章}
\label{\detokenize{list:id4}}

\section{离散数学}
\label{\detokenize{list:id5}}
\sphinxstepscope


\chapter{数据库}
\label{\detokenize{files/Chap1:id1}}\label{\detokenize{files/Chap1::doc}}
\sphinxAtStartPar
数据是数据值、元数据、数据定义和模式(数据属性)的结合。
数据库是指长期储存在计算机内的、集成的、可共享的数据集合。,并可为各种用户共享。
集成是指数据库中的数据按一定的数据模型组织、描述和储存,具有较小的冗余度,较高的数据独立性和易扩展性。
共享是指数据库中的每项数据可以被不同的用户共享。每个用户可以因不同的目的而访问相同的数据,甚至可以同时访问同一数据。
数据库管理系统是介于用户与操作系统之间的一层数据管理软件。其主要功能包括:
\begin{enumerate}
\sphinxsetlistlabels{\arabic}{enumi}{enumii}{}{.}%
\item {} 
\sphinxAtStartPar
数据定义功能。

\item {} 
\sphinxAtStartPar
数据操纵功能。

\item {} 
\sphinxAtStartPar
数据库的运行管理。

\item {} 
\sphinxAtStartPar
数据库的建立和维护功能。
数据库系统 特点:

\end{enumerate}
\begin{itemize}
\item {} 
\sphinxAtStartPar
数据结构化。

\item {} 
\sphinxAtStartPar
数据共享性高,冗余度低。

\item {} 
\sphinxAtStartPar
功能强大,统一管理。
\begin{itemize}
\item {} 
\sphinxAtStartPar
提供事务支持。

\item {} 
\sphinxAtStartPar
保持完整性。

\item {} 
\sphinxAtStartPar
增强安全性。

\end{itemize}

\item {} 
\sphinxAtStartPar
加强了标准化。

\item {} 
\sphinxAtStartPar
等。

\end{itemize}

\sphinxAtStartPar
在历史的条件下阐述数据库的发展和未来的方向。



\renewcommand{\indexname}{索引}
\printindex
\end{document}